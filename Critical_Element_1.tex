\documentclass[]{article}
\usepackage{lmodern}
\usepackage{amssymb,amsmath}
\usepackage{ifxetex,ifluatex}
\usepackage{fixltx2e} % provides \textsubscript
\ifnum 0\ifxetex 1\fi\ifluatex 1\fi=0 % if pdftex
  \usepackage[T1]{fontenc}
  \usepackage[utf8]{inputenc}
\else % if luatex or xelatex
  \ifxetex
    \usepackage{mathspec}
  \else
    \usepackage{fontspec}
  \fi
  \defaultfontfeatures{Ligatures=TeX,Scale=MatchLowercase}
\fi
% use upquote if available, for straight quotes in verbatim environments
\IfFileExists{upquote.sty}{\usepackage{upquote}}{}
% use microtype if available
\IfFileExists{microtype.sty}{%
\usepackage{microtype}
\UseMicrotypeSet[protrusion]{basicmath} % disable protrusion for tt fonts
}{}
\usepackage[margin=1in]{geometry}
\usepackage{hyperref}
\hypersetup{unicode=true,
            pdftitle={Oregon Extended Analyses: 2019},
            pdfauthor={Brock Rowley; Sevrina Tindal; Phillip Irvin; Daniel Anderson; Gerald Tindal},
            pdfborder={0 0 0},
            breaklinks=true}
\urlstyle{same}  % don't use monospace font for urls
\usepackage{longtable,booktabs}
\usepackage{graphicx,grffile}
\makeatletter
\def\maxwidth{\ifdim\Gin@nat@width>\linewidth\linewidth\else\Gin@nat@width\fi}
\def\maxheight{\ifdim\Gin@nat@height>\textheight\textheight\else\Gin@nat@height\fi}
\makeatother
% Scale images if necessary, so that they will not overflow the page
% margins by default, and it is still possible to overwrite the defaults
% using explicit options in \includegraphics[width, height, ...]{}
\setkeys{Gin}{width=\maxwidth,height=\maxheight,keepaspectratio}
\IfFileExists{parskip.sty}{%
\usepackage{parskip}
}{% else
\setlength{\parindent}{0pt}
\setlength{\parskip}{6pt plus 2pt minus 1pt}
}
\setlength{\emergencystretch}{3em}  % prevent overfull lines
\providecommand{\tightlist}{%
  \setlength{\itemsep}{0pt}\setlength{\parskip}{0pt}}
\setcounter{secnumdepth}{0}
% Redefines (sub)paragraphs to behave more like sections
\ifx\paragraph\undefined\else
\let\oldparagraph\paragraph
\renewcommand{\paragraph}[1]{\oldparagraph{#1}\mbox{}}
\fi
\ifx\subparagraph\undefined\else
\let\oldsubparagraph\subparagraph
\renewcommand{\subparagraph}[1]{\oldsubparagraph{#1}\mbox{}}
\fi

%%% Use protect on footnotes to avoid problems with footnotes in titles
\let\rmarkdownfootnote\footnote%
\def\footnote{\protect\rmarkdownfootnote}

%%% Change title format to be more compact
\usepackage{titling}

% Create subtitle command for use in maketitle
\providecommand{\subtitle}[1]{
  \posttitle{
    \begin{center}\large#1\end{center}
    }
}

\setlength{\droptitle}{-2em}

  \title{Oregon Extended Analyses: 2019}
    \pretitle{\vspace{\droptitle}\centering\huge}
  \posttitle{\par}
    \author{Brock Rowley \\ Sevrina Tindal \\ Phillip Irvin \\ Daniel Anderson \\ Gerald Tindal}
    \preauthor{\centering\large\emph}
  \postauthor{\par}
      \predate{\centering\large\emph}
  \postdate{\par}
    \date{6/29/2019}

\pagenumbering{gobble}
\usepackage{placeins}
\usepackage{float}
\usepackage{caption}
\captionsetup[figure]{labelformat = empty}
\usepackage{xcolor}
\definecolor{link}{rgb}{0, 0, 238}
\usepackage{booktabs}

\begin{document}
\maketitle

{
\setcounter{tocdepth}{5}
\tableofcontents
}
\hypertarget{critical-element-1-statewide-system-of-standards-and-assessments}{%
\subsection{Critical Element 1: Statewide System of Standards and
Assessments}\label{critical-element-1-statewide-system-of-standards-and-assessments}}

\hypertarget{state-adoption-of-academic-content-standards-for-all-students}{%
\subsubsection{1.1 State Adoption of Academic Content Standards for All
Students}\label{state-adoption-of-academic-content-standards-for-all-students}}

The Oregon State Board of Education (SBE) adopted new, challenging
academic content standards, the
\color{link}\href{https://www.oregon.gov/ode/educator-resources/standards/Pages/default.aspx}{Common
Core State Standards (CCSS)}\color{black}, in English language arts and
mathematics in Grades K-12 on October 28, 2010. These CCSS are utilized
for all students in Oregon's public schools. Oregon was actively
involved in the development of the CCSS, as the Oregon Department of
Education (ODE), the Educational Enterprise Steering Committee (EESC),
Oregon's Education Service Districts, and school district
representatives provided feedback on the draft CCSS standards.

Similarly, the SBE adopted the
\color{link}\href{https://www.oregon.gov/ode/educator-resources/standards/science/Pages/Science-Standards.aspx}{Next
Generation Science Standards (NGSS)} \color{black} on March 6, 2014. The
NGSS establish learning targets for all students in Oregon's public
schools in Grades K-12. The ODE and the Oregon Science Content and
Assessment Panel provided direct feedback related to the NGSS. The NGSS
are being phased in over time instructionally, so students are being
assessed relative to the Oregon Science (ORSci) standards that were
adopted in 2009.

The newly adopted academic content standards were then reduced in depth,
breadth, and complexity through a process called essentialization. The
new
\color{link}\href{http://www.brtprojects.org/publications/training-modules}{Essentialized
Assessment Frameworks (EAFs)} \color{black} were then used for item
writing for the ORExt. The tables below provide examples of
essentialized standards in grades 5, 8, \& 11 in the subject areas of
English language arts (ELA), mathematics, and science. In the right
column are designations for estimated difficulty of an item: L (low), M
(medium), and H (high). More information on the essentialization process
can be found in section 1.2.

See \emph{Appendix} 1.1 for a User Guide that explains the development
process and intended uses for the EAFs.

\FloatBarrier

\includegraphics{Figures/Standards/Grade5.pdf}

\includegraphics{Figures/Standards/Grade8.pdf}

\includegraphics{Figures/Standards/Grade11.pdf}

\clearpage

\hypertarget{coherent-and-rigorous-academic-content-standards}{%
\subsubsection{1.2 Coherent and Rigorous Academic Content
Standards}\label{coherent-and-rigorous-academic-content-standards}}

The CCSS, ORSci, and NGSS define what students in Oregon should know and
be able to do by the time they graduate from high school. These CCSS,
which were developed by national stakeholders and education experts,
have been determined to be coherent and rigorous by researchers at the
Fordham Institute (see \emph{Appendix} 1.2). They were also developed
with wide stakeholder involvement, particularly here in Oregon. The new
ORExt is linked directly to the content in the CCSS in English language
Arts (reading, writing, \& language) and mathematics. The ORExt is
dually linked to the ORSci as well as the NGSS. The NGSS are widely
accepted by most relevant science instruction organizations as
reflective of rigorous and coherent science concepts.

The new Essentialized Assessment Frameworks (EAFs) are publicly
available. A User Guide is provided to instruct educators regarding the
intended uses of the Essentialized Standards (EsSt), including the
development of Present Levels of Academic Achievement and Functional
Performance (PLAAFP) and Individualized Education Program (IEP) goals
and objectives. The basic essentialization process employed to generate
essentialized standards and write aligned items for the ORExt is
outlined below. The process can also be used to support the development
of curricular and instructional materials, founded in research-based
pedagogy. \FloatBarrier
\includegraphics{Figures/Essentialization/Essentialization.pdf} \newpage

\hypertarget{required-assessments}{%
\subsubsection{1.3 Required Assessments}\label{required-assessments}}

The ORExt assessments were administered in the 2018-19 school year in
ELA and Math in grades 3-8 and grade 11; Science is assessed in grades
5, 8, \& 11. This assessment plan meets the requirements for grade level
assessment in grades 3-8 and once in high school (grades 10-12) for ELA
and Mathematics, while Science is assessed once in the 3-5 grade band,
once in the 6-9 grade band, and once in the 10-12 grade band:

\begin{longtable}[]{@{}lllllll@{}}
\toprule
\textbf{Content Area} & \textbf{Grade 3} & \textbf{Grade 4} &
\textbf{Grade 5} & \textbf{Grade 7} & \textbf{Grade 8} & \textbf{Grade
11}\tabularnewline
\midrule
\endhead
English Language Arts & X & X & X & X & X & X\tabularnewline
Mathematics & X & X & X & X & X & X\tabularnewline
Science & & & X & & X & X\tabularnewline
\bottomrule
\end{longtable}

\hypertarget{policies-for-including-all-students-in-assessments}{%
\subsubsection{1.4 Policies for Including All Students in
Assessments}\label{policies-for-including-all-students-in-assessments}}

Originally, Oregon statute required that all students participate in
statewide assessments, with exceptions allowed for district-approved
parent request for assessment waivers (parent opt-out requests) related
to student disability or religious beliefs (see Oregon Administrative
Rule, OAR § 581-022-0612).

Exception of Students with Disabilities from State Assessment Testing:
(1) For the purposes of this rule a ``student with a disability'' is a
student identified under the Individuals with Disabilities Education
Act, consistent with OAR chapter 581, division 015, or a student with a
disability under Section 504 of the Rehabilitation Act of 1973; (2) A
public agency shall not exempt a student with a disability from
participation in the Oregon State Assessment System or any district wide
assessments to accommodate the student's disability unless the parent
has requested such an exemption.

However, House Bill 2655 established a Student Bill of Rights on January
1, 2016, which permitted parents or adult students to annually opt-out
of Oregon's statewide summative assessments, pursuant to OAR §
581-022-1910.

The Governor published a memorandum for Superintendents, Principals, and
District Test Coordinators related to the change (see \emph{Appendix}
1.4.1).

The expectation that all students in the assessed grades participate,
including students with disabilities, is elaborated clearly and
pervasively across all guidance documents. For example in the Oregon
Test Administration Manual (TAM), where it states that, ``All students
enrolled in grades 3-8 and in high school must take the required Oregon
Statewide Assessments offered at their enrolled grade, including
students re-enrolled in the same grade as in the prior year, unless the
student receives a parent-requested exemption\ldots{}'' (see
\emph{Appendix} 1.4.2, p.~93).

\hypertarget{a-english-learners}{%
\paragraph{1.4A English Learners}\label{a-english-learners}}

English learners are included as appropriate in Oregon's statewide
assessment system. (see \emph{Appendix} 1.4A.1, pp.~31-33). The Smarter
Balanced assessment directions are translated into multiple languages
and available via the Oaks portal. OAR 581-022-0620 (2) requires ODE to
provide translated OAKS assessments for populations at or above 9\% in
grades K-12 within three years after the school year in which the
language exceeds the threshold (see \emph{Appendix} 1.4A.2). In
addition, the accommodations available to students who participate in
the ORExt include translation into the native language, where
appropriate (see \emph{Appendix} 2.3A1, pp.~36-43).

\hypertarget{b-native-language-assessments}{%
\paragraph{1.4B Native Language
Assessments}\label{b-native-language-assessments}}

The ORExt is not administered in a native language format, though it can
be translated into a student's home language.

\hypertarget{participation-data}{%
\subsubsection{1.5 Participation Data}\label{participation-data}}

Oregon's participation data indicate that most students in the tested
grade levels are included in our assessment system. The students with
disabilities subgroup did not meet minimum participation requirements in
2017-18, the most current data available at the time of this report, in
English Language Arts or Mathematics, with rates at 87.5\% and 86.4\%,
respectively. See the table below for a summary of participation.
Documentation of this requirement is provided within the Annual
Performance Report, Indicator B3, which is submitted to the United
States Department of Education's (USED's) Office of Special Education
Programs (OSEP). Participation and performance summaries are provided
below. Additional information regarding state performance is published
in the 2017-18
\color{link}\href{https://www.oregon.gov/ode/schools-and-districts/reportcards/reportcards/Pages/RC-Resource-Archives-1617-2021.aspx}{Statewide
Report Card} \color{black} (see \emph{Appendix} 1.5, pages 1-16 for
student and teacher demographics and pages 38-41 for assessment
information).\FloatBarrier
\includegraphics{Figures/Participation/Participation.png}


\end{document}
