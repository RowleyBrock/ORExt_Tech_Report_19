\documentclass[]{article}
\usepackage{lmodern}
\usepackage{amssymb,amsmath}
\usepackage{ifxetex,ifluatex}
\usepackage{fixltx2e} % provides \textsubscript
\ifnum 0\ifxetex 1\fi\ifluatex 1\fi=0 % if pdftex
  \usepackage[T1]{fontenc}
  \usepackage[utf8]{inputenc}
\else % if luatex or xelatex
  \ifxetex
    \usepackage{mathspec}
  \else
    \usepackage{fontspec}
  \fi
  \defaultfontfeatures{Ligatures=TeX,Scale=MatchLowercase}
\fi
% use upquote if available, for straight quotes in verbatim environments
\IfFileExists{upquote.sty}{\usepackage{upquote}}{}
% use microtype if available
\IfFileExists{microtype.sty}{%
\usepackage{microtype}
\UseMicrotypeSet[protrusion]{basicmath} % disable protrusion for tt fonts
}{}
\usepackage[margin=1in]{geometry}
\usepackage{hyperref}
\hypersetup{unicode=true,
            pdftitle={Oregon Extended Analyses: 2019},
            pdfauthor={Brock Rowley; Sevrina Tindal; Philip Irvin; Daniel Anderson; Gerald Tindal},
            pdfborder={0 0 0},
            breaklinks=true}
\urlstyle{same}  % don't use monospace font for urls
\usepackage{graphicx,grffile}
\makeatletter
\def\maxwidth{\ifdim\Gin@nat@width>\linewidth\linewidth\else\Gin@nat@width\fi}
\def\maxheight{\ifdim\Gin@nat@height>\textheight\textheight\else\Gin@nat@height\fi}
\makeatother
% Scale images if necessary, so that they will not overflow the page
% margins by default, and it is still possible to overwrite the defaults
% using explicit options in \includegraphics[width, height, ...]{}
\setkeys{Gin}{width=\maxwidth,height=\maxheight,keepaspectratio}
\IfFileExists{parskip.sty}{%
\usepackage{parskip}
}{% else
\setlength{\parindent}{0pt}
\setlength{\parskip}{6pt plus 2pt minus 1pt}
}
\setlength{\emergencystretch}{3em}  % prevent overfull lines
\providecommand{\tightlist}{%
  \setlength{\itemsep}{0pt}\setlength{\parskip}{0pt}}
\setcounter{secnumdepth}{0}
% Redefines (sub)paragraphs to behave more like sections
\ifx\paragraph\undefined\else
\let\oldparagraph\paragraph
\renewcommand{\paragraph}[1]{\oldparagraph{#1}\mbox{}}
\fi
\ifx\subparagraph\undefined\else
\let\oldsubparagraph\subparagraph
\renewcommand{\subparagraph}[1]{\oldsubparagraph{#1}\mbox{}}
\fi

%%% Use protect on footnotes to avoid problems with footnotes in titles
\let\rmarkdownfootnote\footnote%
\def\footnote{\protect\rmarkdownfootnote}

%%% Change title format to be more compact
\usepackage{titling}

% Create subtitle command for use in maketitle
\providecommand{\subtitle}[1]{
  \posttitle{
    \begin{center}\large#1\end{center}
    }
}

\setlength{\droptitle}{-2em}

  \title{Oregon Extended Analyses: 2019}
    \pretitle{\vspace{\droptitle}\centering\huge}
  \posttitle{\par}
    \author{Brock Rowley \\ Sevrina Tindal \\ Philip Irvin \\ Daniel Anderson \\ Gerald Tindal}
    \preauthor{\centering\large\emph}
  \postauthor{\par}
      \predate{\centering\large\emph}
  \postdate{\par}
    \date{6/29/2019}

\pagenumbering{gobble}
\usepackage{placeins}
\usepackage{float}
\usepackage{caption}
\captionsetup[figure]{labelformat = empty}
\usepackage{xcolor}
\definecolor{link}{rgb}{0, 0, 238}
\usepackage{booktabs}

\begin{document}
\maketitle

{
\setcounter{tocdepth}{5}
\tableofcontents
}
\hypertarget{critical-element-6---academic-achievement-standards-and-reporting}{%
\subsection{Critical Element 6 - Academic Achievement Standards and
reporting}\label{critical-element-6---academic-achievement-standards-and-reporting}}

\hypertarget{state-adoption-of-alternate-academic-achievement-standards-for-swscd}{%
\subsubsection{6.1 State Adoption of Alternate Academic Achievement
Standards for
SWSCD}\label{state-adoption-of-alternate-academic-achievement-standards-for-swscd}}

The Oregon Extended assessment (ORExt), Oregon's Alternate Assessment
based on Alternate Academic Achievement Standards (AA-AAAS), is part of
the Oregon Statewide Assessment System. The ORExt is administered to
Oregon students with the most significant cognitive disabilities (SWSCD)
in English language arts and mathematics in Grades 3-8 and 11. The ORExt
is administered in science in Grades 5, 8, \& 11. The ORExt links to the
CCSS in English language arts and mathematics. The new ORExt is dually
linked to Oregon's former science standards, as well as to the NGSS.
Results from the English language arts and math administrations are
included in calculations of participation and performance for Annual
Measurable Objectives (AMO) - a provision of the No Child Left Behind
Act (NCLB). Science participation is also included as part of the Title
1 Assessment System requirements, and is administered in grades 5, 8, \&
11. The revised ORExt is built upon a vertical scale in order to support
reliable determinations of annual academic growth in ELA and mathematics
in Grades 3-8. The complete vertical scaling plan and operational item
selection decision rules are located in \emph{Appendix} 2.2.1.

\hypertarget{a-state-formally-adopted-alternate-academic-achievement-standards}{%
\paragraph{6.1A State Formally Adopted Alternate Academic Achievement
Standards}\label{a-state-formally-adopted-alternate-academic-achievement-standards}}

The State Board of Education formally adopted the AAAS and achievement
level descriptors (ALDs) on June 25, 2015 (see \emph{Appendix} 6.1A.1).
The ELA, Math, and Science AAAS, including both the ALDs and the
requisite cut scores are included in \emph{Appendix} 6.1.A.2.

\hypertarget{b-state-applies-aaas-to-all-public-school-swscd-in-tested-grades}{%
\paragraph{6.1B State Applies AAAS to All Public School SWSCD in Tested
Grades}\label{b-state-applies-aaas-to-all-public-school-swscd-in-tested-grades}}

The state applies the AAAS to all public school-served SWSCD who
participate in the ORExt in Grades 3-8 \& 11 in English language arts
and mathematics, and in Grades 5, 8, \& 11 in science.

\hypertarget{c-states-aaas-include-at-least-three-levels-alds-and-cut-scores}{%
\paragraph{6.1C State's AAAS Include At Least Three Levels, ALDs, and
Cut
Scores}\label{c-states-aaas-include-at-least-three-levels-alds-and-cut-scores}}

The alternate academic achievement standards in Oregon are composed of
four levels (though only three are required). In descending order, they
are (a) Level 1, (b) Level 2, (c) Level 3, and (d) Level 4. Level 1 and
Level 2 performances represent proficient achievement, while the bottom
two levels represent achievement that is not yet proficient. The
procedures followed to develop Oregon's alternate academic achievement
standards were consistent with Title 1 assessment system requirements,
including the establishment of cut scores, where relevant. In order to
define four levels of proficiency, Oregon set three cut scores across
all subject areas: (a) to separate Level 1 from Level 2, (b) to separate
Level 2 from Level 3, and, (c) to separate Level 3 from Level 4. The
alternate academic achievement standards in English language arts,
mathematics, and science for the ORExt, including the achievement level
descriptors (ALDs) and cut scores, were established during standard
setting meetings held on June 15 (science), 16 (mathematics), and 17
(English language arts).

\hypertarget{achievement-standard-setting}{%
\subsubsection{6.2 Achievement Standard
Setting}\label{achievement-standard-setting}}

Standard Setting meetings were held at the University of Oregon in
Eugene, OR on June 15, 2015 (Science), June 16, 2015 (Mathematics), and
June 17, 2015 (English language arts). A total of 53 standard setters
were involved in the process: 11 in Science, and 21 in both English
language arts and Mathematics. Panelists were assembled in grade level
teams of three, where two members were special educators and one member
was a content specialist.

The panelists were highly educated. Over 90\% of the panel possessed a
Master's degree or higher. Fifty-seven (57\%) percent of the panelists
had over 11 years of teaching experience. Seventy-six percent (76\%) of
the panelists had some experience working with students with significant
cognitive disabilities with 64\% licensed as Special Educators. The
majority of panel members were female (87\%), from the Northwest of the
state (87\%), and White (83\%). No panel member self-identified with
Oregon's major minority population (Hispanic).

In addition to the live training during standard setting meetings,
panelists were asked to complete several training requirements prior to
the standard setting meetings, which oriented them to the student
population of students with significant cognitive disabilities (SWSCDs),
the Oregon Extended Assessment test design and history, as well as the
bookmarking standard setting method. Panelists were quite confident in
their preparation and final judgments, as evidenced by responses to the
questions: (a) " The training helped me understand the bookmark method
and how to perform my role as a standard setter." (b) ``I am confident
about the defensibility and appropriateness of the final recommended cut
scores.'' and, (c) ``Overall, I am confident that the standard setting
procedures allowed me to use my experience and expertise to recommend
cut scores for the ORExt.'' The hearty majority of standard setters
strongly agreed with these statements, while all participants agreed.

The nine-step process implemented for these standard setting meetings
was based on Hambleton \& Pitoniak (2006) as reported by R.L. Brennan
(Educational Measurement, 4th Edition, pp.~433-470). Standard setting
evaluation questions posed to participants were adapted from Cizek's
Setting Performance Standards (2012). Standard setters set cut scores
and recommended Achievement Level Descriptors (ALDs) for the Oregon
State Board of Education to consider. The cut scores were articulated to
reflect vertical development, or at least maintenance, of expectations
across grades in a manner that respected standard setter judgments to
the greatest possible degree. Six changes were made in ELA and
Mathematics. Science is not built upon a vertical scale, so no cut score
adjustments were necessary in Science. The cut scores are listed below.

\FloatBarrier

\includegraphics{tifs/elacutscore.png}
\includegraphics{tifs/mathcutscore.png}
\includegraphics{tifs/sciencecutscore.png} Note: The ELA and Math
vertical scales for the ORExt are centered on 200 in grades 3-8 and can
be used to document year-to-year growth. None of the other scales should
be used for longitudinal comparisons. All Grade 11 scales are
independent and centered on 900. The grade 5 Science scale is
independent and centered on 500, while the Grade 8 Science scale is
independent and centered on 800. An independent auditor evaluated the
bookmarking standard setting process. The auditor's comprehensive report
can be found in \emph{Appendix} 6.2.2.

\hypertarget{challenging-and-aligned-academic-achievement-standards}{%
\subsubsection{6.3 Challenging and Aligned Academic Achievement
Standards}\label{challenging-and-aligned-academic-achievement-standards}}

Oregon educators initially evaluated new Oregon Essentialized Assessment
Frameworks in two respects. First, educators were asked to determine the
appropriateness of the standards selected for inclusion and exclusion in
the Essentialized Standards (yes/no). Second, the level of linkage
between the Essentialized Standards and grade level content standard was
evaluated (0 = no link, 1 = sufficient link, 2 = strong link). Summary
results are provided in the tables below. A comprehensive essentialized
standard to grade level standard linkage study, as well as essentialized
standard to item alignment study, is provided in \emph{Appendix} 3.1A.

\FloatBarrier

\includegraphics{tifs/alignstudy.png}

\hypertarget{reporting}{%
\subsubsection{6.4 Reporting}\label{reporting}}

Oregon's reporting system facilitates appropriate, credible, and
defensible interpretation and use of its assessment data. With regard to
the ORExt, the purpose is to provide the state technically adequate
student performance data to ascertain proficiency on grade level state
content standards for students with significant cognitive disabilities
(see Sections 3 and 4). In addition, the state makes it clear that
results from the Oregon Extended are not comparable to results from the
SBA/OAKS (see \emph{Appendix} 2.3, p.~29-31). Nevertheless, the test
meets rigorous reliability expectations (see Section 4.1). Validity is
considered here as an overarching summation of the Oregon Extended
assessment system, as well as the mechanisms that Oregon uses to
continuously improve the ORExt assessment (see \emph{Appendix} 2.3B.10).

\hypertarget{a-public-reporting}{%
\paragraph{6.4A Public Reporting}\label{a-public-reporting}}

Oregon reports participation and assessment results for all students and
for each of the required subgroups in its reports at the school,
district, and state levels. The state does not report subgroup results
when these results would reveal personally identifiable information
about an individual student. The calculation rule followed is that the
number of students in the subgroup must meet the minimum cell size
requirement for each AMO decision: participation, achievement in English
language arts and math, attendance, and graduation, where appropriate
(see \emph{Appendix} 2.6C).

\hypertarget{b-state-reports-interpretable-results}{%
\paragraph{6.4B State Reports Interpretable
Results}\label{b-state-reports-interpretable-results}}

Oregon develops and disseminates individual student data upon final
determination of accuracy. The state provides districts with individual
student reports (ISRs) that meet most relevant requirements. The state
incorporated the Standard Error of Measure (SEM) for each student score
into the report templates. The SEM associated with each cut score is
provided in Section 4.1B. Also, see the mock-up ISR in \emph{Appendix}
6.4C.

\hypertarget{c1---c5-state-provides-individual-student-reports}{%
\paragraph{6.4C1 - C5 State Provides Individual Student
Reports}\label{c1---c5-state-provides-individual-student-reports}}

Oregon's student reports provide valid and reliable information
regarding achievement on the assessments relative to the AAS. The
reliability of the data is addressed in Section 4.1. Validity is
considered here as an overarching summation of the Oregon Extended
assessment system, as well as the mechanisms that Oregon uses to
continuously improve the Oregon Extended assessment. The ISRs clearly
demonstrate the students' scale score relative the AAAS that is relevant
for that content area and grade level (see Section 4.4 and
\emph{Appendix} 6.4C). The Oregon ISRs provide information for parents,
teachers, and administrators to help them understand and address a
student's academic needs. These reports are displayed in a simple format
that is easy for stakeholders to understand. District representatives
can translate results for parents as necessary. Scaled score
interpretation guidance is published in \emph{Appendix} 2.1A.

\hypertarget{conclusions-and-next-steps}{%
\subsection{Conclusions and Next
Steps}\label{conclusions-and-next-steps}}

In sum, the rigor of the procedural development and statistical outcomes
of the ORExt were substantive and support the assessments intended
purpose. Procedural evidence includes essentialized standards
development, item development, item content and bias reviews, an
independent alignment study and item selection based upon item
characteristics. Outcome-related evidence included measure reliability
analyses, point measure biserials, outfit mean squares, item difficulty
and person ability distributions, and convergent and divergent validity
evidence. These sources of evidence were all quite good and provide
important validity evidence.

The test development process adhered to procedural guidelines defined by
the AERA/APA/NCME Standards for Educational and Psychological Testing
(2014), as well as incorporating procedures that are known in the field
to be best practice. For example, an independent auditor evaluated
alignment in 2016-17. Documentation collected in the alignment study
report suggests that the ORExt assessment system is aligned based on
five evaluation components: a) standard selection for essentialization,
b) strength of linkage between essentialized standards and grade level
content standards, c) alignment between items and essentialized
standards, d) alignment between the essentialized standards and the
achievement level descriptors, and e) alignment between the achievement
level descriptors and the ORExt test items. In addition, the ORExt
reflects what highly qualified Oregon educators believe represents the
highest professional standards for the population of students with
significant cognitive disabilities, as evidenced in our consequential
validity study by teacher support of the academic content on the ORExt
as well as the behaviors sampled during test administration.

The test reliabilities for the ORExt were quite high, suggesting that
the assessment items functioned consistently with the test as a whole.
The correlations between students' content scores across subjects were
not overly strong, implying that each test measures a distinct
construct. The classification consistency analyses demonstrate that the
ORExt is appropriately categorizing students into the proficient
category, and capable of doing so in a consistent manner. The vertical
scale developed in 2014-15 appears to be modeling incremental growth
across Grades 3-8 in ELA and mathematics, as intended. The Grade 7
mathematics test demonstrated sufficient item difficulties across the
ranges medium and high item complexity. However, low level items must
again be amended in the 2018-19 school year. The ELA and science
assessments could continue to benefit from the addition of more
difficult items, as evidenced by comparisons of the average person
abilities and item difficulties. Mathematics assessments appear to be
functioning quite well in terms of person abilities and item
difficulties.

The Oregon Observational Rating Assessment (ORora) results demonstrate
that approximately 17-25\% of the SWSCD who participated in the ORExt
also took the ORora, depending upon grade level. A total of 529 students
were administered the ORora in 2017-2018 school year. The participants
were primarily students with multiple, severe disabilities with very
limited communication systems. Analyses of missing data patterns for the
ORExt demonstrated that QAs were generally able to adhere to the
discontinuation rules. Quantitative results indicate that a total of 529
students across all tested grades were administered the ORora. Response
patterns on the ORExt were compared to ORora results to determine what
percentages of QAs were administering the ORora due to the minimum
participation rule and what percentage were administering the ORora of
their own volition. Analyses showed that 480 students were eligible to
take the ORora in English language arts, 466 students were eligible to
take the ORora in mathematics, and 86 were eligible to take the ORora in
science. This means that about 30 students per grade, per content area
received five or fewer correct responses within the first 15 items
administered on the ORExt. Of the 600 test records that met ORora
eligibility requirements, 71 were not administered the ORora. In
addition, there were 62 students in ELA and Math, respectively, who were
administered the ORora without having participated in the ORExt (54 of
those students were the same students, across each content area, with
eight students unique to each content area, respectively).

The 2017-18 Oregon Consequential Validity study provides important
information for future administrations of the ORExt. Results indicate
historical concerns that are not possible to address, such as the
ongoing tension between assessing life skills and academics, but also to
some actionable steps with a focus toward continuous improvement.
Respondents pointed to positive attributes of the ORExt, especially
those involving test administration and design and felt somewhat
positive regarding various educational impacts of the ORExt.

During the 2017-18 ORExt testing window, feedback from the field and the
number of students administered the tablet based ORExt indicated
assessors preferred administration of the tablet/web-based assessment
versus paper/pencil. Benefits expressed by the field indicated increased
student engagement, improved standardization, ease of use by teachers,
and resource protection (i.e., time, printing, expense). Practice tests
were available to familiarize teachers and students to the tablet format
prior to administration of the secure tests. Based on the 2017-18
testing window, enhancements are in process to improve the
tablet/web-based administration for the 2018-19 testing window. These
improvements include updates to make administration/data entry more
efficient for assessors and additional alerts if devices are no longer
online. The 2018-19 testing window will be the first year all data entry
will be held on the BRT servers. ODE will no longer provide a
paper/pencil data entry platform.

Documenting evidence of validity remains an ongoing and continuous
process. Our efforts to continue to improve the assessment system are
outlined below, as well as in Sections 3 and 4 above. We also have
studies planned over the course of the next three years that will help
to solidify the evidence that is accumulating. All of the evidence we
have at hand suggests that the ORExt is sufficient to its stated purpose
of providing reliable determinations of student proficiency at the test
level in order to support systems level analysis of district and state
programs. The ORExt will hopefully continue to improve over time due to
field-testing and constant monitoring and review, and additional
validity evidence will be gathered.

As mentioned above in Section 3.1A, data are presented to support the
claim that Oregon's AA-AAAS provides the state technically adequate
student performance data to ascertain proficiency on grade level state
content standards for students with significant cognitive disabilities -
which is its defined purpose. In this technical report, we have provided
content validity evidence related to the ORExt test development process
(i.e., essentialization process, linkage study, distributed item review,
test blueprint, item writer training and demographics, and item reviewer
training and demographics), ORExt test reliability evidence, and ORExt
consequential validity evidence. Further analyses over the coming years
are planned to continue the development of technical documentation for
overall construct validity of the ORExt. The technical documentation
plan for the 2017 through 2019 school years is provided below:

\FloatBarrier

\includegraphics{tifs/techdocplan.png}


\end{document}
