\documentclass[]{article}
\usepackage{lmodern}
\usepackage{amssymb,amsmath}
\usepackage{ifxetex,ifluatex}
\usepackage{fixltx2e} % provides \textsubscript
\ifnum 0\ifxetex 1\fi\ifluatex 1\fi=0 % if pdftex
  \usepackage[T1]{fontenc}
  \usepackage[utf8]{inputenc}
\else % if luatex or xelatex
  \ifxetex
    \usepackage{mathspec}
  \else
    \usepackage{fontspec}
  \fi
  \defaultfontfeatures{Ligatures=TeX,Scale=MatchLowercase}
\fi
% use upquote if available, for straight quotes in verbatim environments
\IfFileExists{upquote.sty}{\usepackage{upquote}}{}
% use microtype if available
\IfFileExists{microtype.sty}{%
\usepackage{microtype}
\UseMicrotypeSet[protrusion]{basicmath} % disable protrusion for tt fonts
}{}
\usepackage[margin=1in]{geometry}
\usepackage{hyperref}
\hypersetup{unicode=true,
            pdftitle={Oregon Extended Analyses: 2019},
            pdfauthor={Brock Rowley; Sevrina Tindal; Philip Irvin; Daniel Anderson; Gerald Tindal},
            pdfborder={0 0 0},
            breaklinks=true}
\urlstyle{same}  % don't use monospace font for urls
\usepackage{graphicx,grffile}
\makeatletter
\def\maxwidth{\ifdim\Gin@nat@width>\linewidth\linewidth\else\Gin@nat@width\fi}
\def\maxheight{\ifdim\Gin@nat@height>\textheight\textheight\else\Gin@nat@height\fi}
\makeatother
% Scale images if necessary, so that they will not overflow the page
% margins by default, and it is still possible to overwrite the defaults
% using explicit options in \includegraphics[width, height, ...]{}
\setkeys{Gin}{width=\maxwidth,height=\maxheight,keepaspectratio}
\IfFileExists{parskip.sty}{%
\usepackage{parskip}
}{% else
\setlength{\parindent}{0pt}
\setlength{\parskip}{6pt plus 2pt minus 1pt}
}
\setlength{\emergencystretch}{3em}  % prevent overfull lines
\providecommand{\tightlist}{%
  \setlength{\itemsep}{0pt}\setlength{\parskip}{0pt}}
\setcounter{secnumdepth}{0}
% Redefines (sub)paragraphs to behave more like sections
\ifx\paragraph\undefined\else
\let\oldparagraph\paragraph
\renewcommand{\paragraph}[1]{\oldparagraph{#1}\mbox{}}
\fi
\ifx\subparagraph\undefined\else
\let\oldsubparagraph\subparagraph
\renewcommand{\subparagraph}[1]{\oldsubparagraph{#1}\mbox{}}
\fi

%%% Use protect on footnotes to avoid problems with footnotes in titles
\let\rmarkdownfootnote\footnote%
\def\footnote{\protect\rmarkdownfootnote}

%%% Change title format to be more compact
\usepackage{titling}

% Create subtitle command for use in maketitle
\providecommand{\subtitle}[1]{
  \posttitle{
    \begin{center}\large#1\end{center}
    }
}

\setlength{\droptitle}{-2em}

  \title{Oregon Extended Analyses: 2019}
    \pretitle{\vspace{\droptitle}\centering\huge}
  \posttitle{\par}
    \author{Brock Rowley \\ Sevrina Tindal \\ Philip Irvin \\ Daniel Anderson \\ Gerald Tindal}
    \preauthor{\centering\large\emph}
  \postauthor{\par}
      \predate{\centering\large\emph}
  \postdate{\par}
    \date{6/29/2019}

\pagenumbering{gobble}
\usepackage{placeins}
\usepackage{float}
\usepackage{caption}
\captionsetup[figure]{labelformat = empty}
\usepackage{xcolor}
\definecolor{link}{rgb}{0, 0, 238}
\usepackage{booktabs}

\begin{document}
\maketitle

{
\setcounter{tocdepth}{5}
\tableofcontents
}
\hypertarget{appendix-descriptions}{%
\subsection{Appendix Descriptions}\label{appendix-descriptions}}

\hypertarget{appendix-1.1}{%
\paragraph{Appendix 1.1}\label{appendix-1.1}}

\emph{Appendix} 1.1 explains the development process and intended uses
for the Essentialized Assessment Frameworks (EAFs). The EAFs are the
essentialized standards (EsSt), which are linked to grade level content
standards. The ORExt is aligned to the EAFs, as well. While the EAFs
primarily guide item development, they are also intended to be used in
the development of appropriate Present Levels of Functional and Academic
Performance (PLAAFP) statements and Individualized Education Program
(IEP) goals and objectives.

\hypertarget{appendix-1.2}{%
\paragraph{Appendix 1.2}\label{appendix-1.2}}

\emph{Appendix} 1.2 conveys the evaluation conducted by researchers at
the Fordham Institute, which compared then-current state standards to
the CCSS in terms of rigor. The findings generally show that the CCSS
are as rigorous or more rigorous than state standards.

\hypertarget{appendix-1.4.1}{%
\paragraph{Appendix 1.4.1}\label{appendix-1.4.1}}

\emph{Appendix} 1.4.1 is the Executive Memo from the Governor of Oregon
regarding parent opt-out expectations.

\hypertarget{appendix-1.4.2}{%
\paragraph{Appendix 1.4.2}\label{appendix-1.4.2}}

\emph{Appendix} 1.4.2 is the test administration manual (TAM) for all
assessments in the Oregon statewide assessment system, including the
SBA, OAKS, the ORExt, the Kindergarten Assessment, and the ELPA. The TAM
elaborates all relevant test security and administration procedures.

\hypertarget{appendix-1.4a.1}{%
\paragraph{Appendix 1.4A.1}\label{appendix-1.4a.1}}

\emph{Appendix} 1.4A.1 is ODE's English Learner Program Guide, outlining
English learner (EL) system requirements in the areas of student
identification, services, reporting, and assessment for ELs in Oregon's
public schools, including ELs who are SWD.

\hypertarget{appendix-1.4a.2}{%
\paragraph{Appendix 1.4A.2}\label{appendix-1.4a.2}}

\emph{Appendix} 1.4A.2 is Oregon's regulations that require ODE to
provide translated OAKS assessments for populations at or above 9\% in
grades K-12 within three years after the school year in which the
language exceeds the threshold.

\hypertarget{appendix-1.5}{%
\paragraph{Appendix 1.5}\label{appendix-1.5}}

\emph{Appendix} 1.5 is Oregon's annual report to the state legislature
for the 2017-18 school year. The report includes student demographics
and information on student groups, school funding and staff information,
test results, graduation and drop out rates, charter school data and
information on alternative education programs, early childhood data, and
attendance and chronic absenteeism data.

\hypertarget{appendix-2.1}{%
\paragraph{Appendix 2.1}\label{appendix-2.1}}

\emph{Appendix} 2.1 is the test specifications document that describes
our approach to assessment and test design for the ORExt. The document
includes our approach to RDBC, an overview of the essentialization
process and EAF documents, the anticipated operational test design for
the ORExt, test development considerations, sample test items, item
specifications, and universal tools/designated supports/accommodations.

\hypertarget{appendix-2.1a}{%
\paragraph{Appendix 2.1A}\label{appendix-2.1a}}

\emph{Appendix} 2.1A provides the field with comprehensive information
related to scaled score interpretation for the ORExt. The guidance is
published in three main areas: 1) Annual performance, 2) Annual growth,
and 3) Performance for very low functioning students. Guidance regarding
use and interpretation of reading and writing subscores is also
provided.

\hypertarget{appendix-2.1b}{%
\paragraph{Appendix 2.1B}\label{appendix-2.1b}}

\emph{Appendix} 2.1B is the test blueprint for the ORExt, conveying the
balance of representation of domains across the content areas and grade
levels assessed. Operational items are selected to reflect the
representation percentages included in the test blueprint.

\hypertarget{appendix-2.1c}{%
\paragraph{Appendix 2.1C}\label{appendix-2.1c}}

\emph{Appendix} 2.1C describes the eight-step item development process
used to develop items for the ORExt, from standard selection to test
booklet formation. The item development process is specific and explicit
in order to increase transparency.

\hypertarget{appendix-2.2.1}{%
\paragraph{Appendix 2.2.1}\label{appendix-2.2.1}}

\emph{Appendix} 2.2.1 is the set of PPT slides that were used to train
item writers for the ORExt. Item writers were also provided an
orientation to the test specifications as part of training.

\hypertarget{appendix-2.2.2}{%
\paragraph{Appendix 2.2.2}\label{appendix-2.2.2}}

\emph{Appendix} 2.2.2 is a document that summarizes the balanced design
vertical scaling plan employed for the ORExt in the 2014-15
administration. The document includes the domain sampling plan for all
assessments, as well as the decision rules employed to remove items from
the operational item pool prior to vertical scaling and standard setting
procedures.

\hypertarget{appendix-2.2.3}{%
\paragraph{Appendix 2.2.3}\label{appendix-2.2.3}}

\emph{Appendix} 2.2.3 provides stakeholders with visual representation
of the structure of the ORExt. Sample items are conveyed in English
language arts, mathematics, and science, with the scoring protocol and
student materials presented together. Stakeholders can see the structure
of each item, as well as how the items are scored. They can also gather
an idea about the types of formats that are used for answer choices that
are included within the student materials documents.

\hypertarget{appendix-2.3}{%
\paragraph{Appendix 2.3}\label{appendix-2.3}}

\emph{Appendix} 2.3 is ODE's General Administration and Scoring Manual
for 2017-18. The manual establishes ODE's expectations regarding the
test window, utilizing the ORExt training and proficiency website, using
the sign language interpreter training and proficiency website, and
informing parents. It also provides the following information for
stakeholders, including educators and parents: Overview of the Extended
Assessments, Assessing a Student, Scoring, Decision Making, and
Information for Teachers. The manual provides three appendices that
provide guidance regarding the provision of supports, parent questions
and answers, and a glossary.

\hypertarget{appendix-2.3a.1}{%
\paragraph{Appendix 2.3A.1}\label{appendix-2.3a.1}}

\emph{Appendix} 2.3A.1 is the 2017-18 accessibility options manual for
all assessments in the Oregon statewide assessment system, including the
SBA, OAKS, the ORExt, and the ELPA. Options include Universal Tools,
Designated Supports, and Accommodations. The manual provides guidance
regarding use of these options in instruction and assessment, as well as
implementation strategies and use evaluation. Each accommodation is
coded for use in data analysis related to assessment scores for the SBA
and OAKS.

\hypertarget{appendix-2.3a.2}{%
\paragraph{Appendix 2.3A.2}\label{appendix-2.3a.2}}

\emph{Appendix} 2.3A.2 is ODE's How to Select, Administer, and Evaluate
Accommodations on Oregon's Statewide Assessment manual for 2013-14. The
manual trains users regarding how to implement and evaluate appropriate
accommodations, from the student level to the systems level.

\hypertarget{appendix-2.3a.3}{%
\paragraph{Appendix 2.3A.3}\label{appendix-2.3a.3}}

\emph{Appendix} 2.3A.3 is a document that summarizes the procedures used
during item development to reduce item depth, breadth, and complexity,
in addition to the test specifications information found in
\emph{Appendix} 2.1. The document also provides more detail regarding
how language complexity is addressed and reviewed in an effort to
decrease the language load of items and make the test more accessible to
all students. The document also discusses ways in which bias is
addressed during test development.

\hypertarget{appendices-2.3b.1-2.3b.2}{%
\paragraph{Appendices 2.3B.1-2.3B.2}\label{appendices-2.3b.1-2.3b.2}}

Appendices 2.3B.1 and 2.3B.2 are the PowerPoint (PPT) trainings that
were used by ODE and BRT trainers to train new qualified assessors (QAs)
and qualified trainers (QTs) in four regionally hosted trainings in
November 2017. QTs also used the package to train New Qualified
Assessors for the 2017-18 school year. The training provides
participants with the information needed to pass proficiency tests as
part of the requirements to become a QA for the Oregon Extended
Assessments and was delivered by QTs throughout the state. The training
package addresses the following topics: ``What's new in 2017-18?'',
``2018 Test Window'', ``Eligibility - which students take AA-AAAS?'',
``Test administration'', ``Student Confidentiality \& Test Security'',
``Test Administration (Physical \& Logistic)'', ``Scoring \& Data
Entry'', ``Reports \& Sharing Results with Parents'', ``Navigating the
Training and Proficiency website'', and ``Resources.''

\hypertarget{appendix-2.3b.4}{%
\paragraph{Appendix 2.3B.4}\label{appendix-2.3b.4}}

\emph{Appendix} 2.3B.4 is the test calendar for the entire Oregon
statewide assessment program, including the SBA, OAKS, the ORExt, the
ELPA, the Kindergarten Assessment, and the NAEP.

\hypertarget{appendix-2.3b.5}{%
\paragraph{Appendix 2.3B.5}\label{appendix-2.3b.5}}

\emph{Appendix} 2.3B.5 is a sample agenda that ODE makes available to
QTs around the state to train their respective new QAs as they implement
the train-the-trainers model used by the Oregon Extended assessment.

\hypertarget{appendix-2.3b.6}{%
\paragraph{Appendix 2.3B.6}\label{appendix-2.3b.6}}

\emph{Appendix} 2.3B.6 is the list of instructions provided to new QAs
and QTs regarding how to access the online training and proficiency
website.

\hypertarget{appendix-2.3b.7}{%
\paragraph{Appendix 2.3B.7}\label{appendix-2.3b.7}}

\emph{Appendix} 2.3B.7 is the list of responsibilities associated with
being a QT for the ORExt assessment.

\hypertarget{appendix-2.3b.8}{%
\paragraph{Appendix 2.3B.8}\label{appendix-2.3b.8}}

\emph{Appendix} 2.3B.8 is the document that contains the most commonly
fielded questions and answers from stakeholders, including parents and
teachers.

\hypertarget{appendix-2.3b.9}{%
\paragraph{Appendix 2.3B.9}\label{appendix-2.3b.9}}

\emph{Appendix} 2.3B.9 is the Helpdesk log report that summarizes all of
the technical assistance questions garnered from the field this year.
Efforts are made to find any patterns that our team may use to improve
training for the following year.

\hypertarget{appendix-2.3b.10}{%
\paragraph{Appendix 2.3B.10}\label{appendix-2.3b.10}}

\emph{Appendix} 2.3B.10 is the consequential validity report for the
spring 2017 consequential validity study conducted by BRT. The report
provides documentation of the perceptions in the field related to both
intended and unintended academic and social consequences of the ORExt.

\hypertarget{appendices-2.6}{%
\paragraph{Appendices 2.6}\label{appendices-2.6}}

\emph{Appendix} 2.6 is the data entry guide. The guide explains the
paper/pencil data entry process located on ODE's secure server.

\hypertarget{appendices-2.6a}{%
\paragraph{Appendices 2.6A}\label{appendices-2.6a}}

\emph{Appendix} 2.6A is the ORExt Test Application User Guide. With
2017-18 the first year the tablet/web-based platform was available for
all grade level and subject area tests, this guide walked through the
system requirements, download/login instructions, testing process, and
troubleshooting.

\hypertarget{appendix-2.6c}{%
\paragraph{Appendix 2.6C}\label{appendix-2.6c}}

\emph{Appendix} 2.6C is the manual defining the state of Oregon's
policies and procedures regarding how students are included in AMO
reporting, including how achievement, growth, and graduation rates are
reported for student groups and subgroups.

\hypertarget{appendix-3.1a}{%
\paragraph{Appendix 3.1A}\label{appendix-3.1a}}

\emph{Appendix} 3.1A is a document that summarizes the independent
alignment study process and participants used to review the linkage
between the Essentialized Standards and grade level content standards
(CCSS in ELA and Math; ORSci and NGSS in Science), as well as the
alignment between test items for the ORExt with those Essentialized
Standards. In addition, reviewers rated the items for potential bias and
access concerns. All data was gathered using the Distributed Item Review
(DIR) website, supported by a webinar training and ongoing technical
assistance. The results of the 2014-15 Linkage Study, which was not
independent but run by BRT researchers, are also included.

\hypertarget{appendix-3.1b}{%
\paragraph{Appendix 3.1B}\label{appendix-3.1b}}

\emph{Appendix} 3.1B is a document that describes the Distributed Item
Review (DIR) website used by Oregon teachers to evaluate the alignment
between test items for the ORExt with Essentialized Standards. In
addition, reviewers rated the items for potential bias and access
concerns. All data was gathered using the DIR website, supported by a
webinar training and ongoing technical assistance.

\hypertarget{appendices-4.1}{%
\paragraph{Appendices 4.1}\label{appendices-4.1}}

\emph{Appendix} 4.1 is the Inter-rater Reliability Study Observation
form completed by study participants.

\hypertarget{appendix-4.1b}{%
\paragraph{Appendix 4.1B}\label{appendix-4.1b}}

\emph{Appendix} 4.1B conveys the historical development of the ORExt
from 1999 to the present, including the grade levels/bands assessed,
content areas assessed, and the targeted content standards.

\hypertarget{appendix-4.2}{%
\paragraph{Appendix 4.2}\label{appendix-4.2}}

\emph{Appendix} 4.2 includes the most current published state level data
regarding Oregon's ethnic diversity.

\hypertarget{appendix-5.1b}{%
\paragraph{Appendix 5.1B}\label{appendix-5.1b}}

\emph{Appendix} 5.1B is the revised and rigorous guidance that ODE has
provided to IEP teams to assist them in making appropriate assessment
eligibility determinations for students with disabilities.

\hypertarget{appendix-5.1d}{%
\paragraph{Appendix 5.1D}\label{appendix-5.1d}}

\emph{Appendix} 5.1D includes a summary report of the statewide results
and the administration and scoring instructions for the new Oregon
Observational Rating Assessment (ORora). The ORora is administered to
all students whose ORExt testing was discontinued. It provides
information regarding student progress in terms of functional skills in
adaptive and communication domains for the small subgroup of students
who are unable to meet the academic expectations in the ORExt.

\hypertarget{appendix-6.1a.1}{%
\paragraph{Appendix 6.1A.1}\label{appendix-6.1a.1}}

\emph{Appendix} 6.1A.1 is the agenda and minutes that document the
hearing and adoption of the AAAS for the ORExt on June 25, 2015.

\hypertarget{appendix-6.1a.2}{%
\paragraph{Appendix 6.1A.2}\label{appendix-6.1a.2}}

\emph{Appendix} 6.1A.2 includes all of the achievement level descriptors
(ALDs) and cutscores that define performance for the ORExt in
qualitative and quantitative fashions, respectively. These Alternate
Academic Achievement Standards (AAAS) describe what students should know
and be able to do based upon their performance on the ORExt.

\hypertarget{appendix-6.2.1}{%
\paragraph{Appendix 6.2.1}\label{appendix-6.2.1}}

\emph{Appendix} 6.2.1 is the PPT slides used to train standard setters
during the June 2015 standard setting meetings for ELA, math, and
science.

\hypertarget{appendix-6.2.2}{%
\paragraph{Appendix 6.2.2}\label{appendix-6.2.2}}

\emph{Appendix} 6.2.2 is a standard setting report generated by an
independent auditor. The report provides a comprehensive evaluation of
the bookmark standard setting procedure employed for the ORExt on June
15-17, 2015.

\hypertarget{appendix-6.4c}{%
\paragraph{Appendix 6.4C}\label{appendix-6.4c}}

\emph{Appendix} 6.4C is a document that displays the individual student
report (ISR) that ODE publishes for students who participate in the
ORExt. The mock-up includes cut scores and achievement level descriptors
(ALDs), as well as links to the ODE website for additional information.


\end{document}
