\documentclass[]{article}
\usepackage{lmodern}
\usepackage{amssymb,amsmath}
\usepackage{ifxetex,ifluatex}
\usepackage{fixltx2e} % provides \textsubscript
\ifnum 0\ifxetex 1\fi\ifluatex 1\fi=0 % if pdftex
  \usepackage[T1]{fontenc}
  \usepackage[utf8]{inputenc}
\else % if luatex or xelatex
  \ifxetex
    \usepackage{mathspec}
  \else
    \usepackage{fontspec}
  \fi
  \defaultfontfeatures{Ligatures=TeX,Scale=MatchLowercase}
\fi
% use upquote if available, for straight quotes in verbatim environments
\IfFileExists{upquote.sty}{\usepackage{upquote}}{}
% use microtype if available
\IfFileExists{microtype.sty}{%
\usepackage{microtype}
\UseMicrotypeSet[protrusion]{basicmath} % disable protrusion for tt fonts
}{}
\usepackage[margin=1in]{geometry}
\usepackage{hyperref}
\hypersetup{unicode=true,
            pdftitle={Oregon Extended Analyses: 2019},
            pdfauthor={Brock Rowley; Sevrina Tindal; Phillip Irvin; Daniel Anderson; Gerald Tindal},
            pdfborder={0 0 0},
            breaklinks=true}
\urlstyle{same}  % don't use monospace font for urls
\usepackage{graphicx,grffile}
\makeatletter
\def\maxwidth{\ifdim\Gin@nat@width>\linewidth\linewidth\else\Gin@nat@width\fi}
\def\maxheight{\ifdim\Gin@nat@height>\textheight\textheight\else\Gin@nat@height\fi}
\makeatother
% Scale images if necessary, so that they will not overflow the page
% margins by default, and it is still possible to overwrite the defaults
% using explicit options in \includegraphics[width, height, ...]{}
\setkeys{Gin}{width=\maxwidth,height=\maxheight,keepaspectratio}
\IfFileExists{parskip.sty}{%
\usepackage{parskip}
}{% else
\setlength{\parindent}{0pt}
\setlength{\parskip}{6pt plus 2pt minus 1pt}
}
\setlength{\emergencystretch}{3em}  % prevent overfull lines
\providecommand{\tightlist}{%
  \setlength{\itemsep}{0pt}\setlength{\parskip}{0pt}}
\setcounter{secnumdepth}{0}
% Redefines (sub)paragraphs to behave more like sections
\ifx\paragraph\undefined\else
\let\oldparagraph\paragraph
\renewcommand{\paragraph}[1]{\oldparagraph{#1}\mbox{}}
\fi
\ifx\subparagraph\undefined\else
\let\oldsubparagraph\subparagraph
\renewcommand{\subparagraph}[1]{\oldsubparagraph{#1}\mbox{}}
\fi

%%% Use protect on footnotes to avoid problems with footnotes in titles
\let\rmarkdownfootnote\footnote%
\def\footnote{\protect\rmarkdownfootnote}

%%% Change title format to be more compact
\usepackage{titling}

% Create subtitle command for use in maketitle
\providecommand{\subtitle}[1]{
  \posttitle{
    \begin{center}\large#1\end{center}
    }
}

\setlength{\droptitle}{-2em}

  \title{Oregon Extended Analyses: 2019}
    \pretitle{\vspace{\droptitle}\centering\huge}
  \posttitle{\par}
    \author{Brock Rowley \\ Sevrina Tindal \\ Phillip Irvin \\ Daniel Anderson \\ Gerald Tindal}
    \preauthor{\centering\large\emph}
  \postauthor{\par}
      \predate{\centering\large\emph}
  \postdate{\par}
    \date{6/29/2019}

\pagenumbering{gobble}
\usepackage{placeins}
\usepackage{float}
\usepackage{caption}
\captionsetup[figure]{labelformat = empty}
\usepackage{xcolor}
\definecolor{link}{rgb}{0, 0, 238}
\usepackage{booktabs}

\begin{document}
\maketitle

{
\setcounter{tocdepth}{5}
\tableofcontents
}
\hypertarget{critical-element-5---inclusion-of-all-students}{%
\subsection{Critical Element 5 - Inclusion of All
Students}\label{critical-element-5---inclusion-of-all-students}}

\hypertarget{procedures-for-including-swds}{%
\subsubsection{5.1 Procedures for Including
SWDs}\label{procedures-for-including-swds}}

The Oregon assessment system provides explicit guidance regarding the
participation of all public school students in its statewide assessment
program (see Section 1.4).

\hypertarget{a-clear-explanations-of-the-differences-between-assessments}{%
\paragraph{5.1A Clear Explanations of the Differences Between
Assessments}\label{a-clear-explanations-of-the-differences-between-assessments}}

The assessment options for all public school students in Oregon are
elaborated in the Oregon Test Administration Manual (see \emph{Appendix}
1.4.2, p.~7). These options include the Smarter Balanced Assessment in
English language arts and mathematics in Grades 3-8 \& 11, the Oregon
Assessment of Knowledge and Skills in science in Grades 5, 8, \& 11, and
in the same content areas and grade levels for SWSCD who take the ORExt
(see \emph{Appendix} 1.4.2, p.~92-93). Social studies assessment is a
district option within the OAKS portal, as well. In addition,
expectations for the English Language Proficiency Assessment (ELPA) and
the Kindergarten Assessment are provided.

\hypertarget{b-eligibility-decisions-made-by-iep-teams}{%
\paragraph{5.1B Eligibility Decisions Made by IEP
Teams}\label{b-eligibility-decisions-made-by-iep-teams}}

A student's IEP team determines how a student with disabilities will
participate in the Oregon Statewide Assessment program. The IEP team
must address the eligibility criteria for participation in the ORExt
before determining that the assessment is the appropriate option (see
\emph{Appendix} 5.1B).

\hypertarget{c-guidelines-for-assessment-selection}{%
\paragraph{5.1C Guidelines for Assessment
Selection}\label{c-guidelines-for-assessment-selection}}

As noted earlier, IEP teams make decisions regarding how students with
disabilities participate in the Oregon statewide assessment program. At
present, students participate in one of three options: (a) student takes
the general assessment with or without universal tools. (b) student
takes the general assessment with designated supports and/or
accommodations, or (c) student takes the ORExt. Guidelines for making
universal support, designated support, and accommodations decisions for
the general assessments are provided in \emph{Appendix} 2.3A.1.
Guidelines for making these determinations for SWSCD who participate in
AA-AAAS are provided in \emph{Appendix} 5.1B.

\hypertarget{d-information-on-accessibility-options}{%
\paragraph{5.1D Information on Accessibility
Options}\label{d-information-on-accessibility-options}}

Information regarding accessibility options for the general assessment
can be found with the general assessment Peer Review evidence. For the
ORExt, accessibility is treated holistically, with universal design for
assessment concepts embedded in the item design and a wide variety of
accommodations also available if needed. Items are crafted to be
visually simple and clean. Graphic supports, which are always
black/white line drawings, are embedded in all items at the low level of
complexity but are phased out as items become more complex. Items are
designed to incorporate simplified language unless specific academic
vocabulary and concepts is what is being tested (see \emph{Appendix}
2.3A.3). The items on the ORExt are all selected response, with three
response options allowing for multiple modes of access (e.g., saying the
answer, pointing to the answer, eye gaze, switch, etc.). All text
presented to students is at least 18-pt font (larger, of course, in the
large print version). Sample items are presented in \emph{Appendix}
2.2.3. All accessibility supports, designated supports, and
accommodations for the ORExt are published in \emph{Appendix} 2.3A.1,
p.~36-43. For students who have very limited to no communication and are
unable to access even the most accessible items on the ORExt, an Oregon
Observational Rating Assessment (ORora) was implemented in 2015-16. The
ORora is completed by teachers and documents the student's level of
communication complexity (expressive and receptive), as well as level of
independence in the domains of attention/joint attention and
mathematics. The administration instructions and 2017-18 results for the
ORora are included in \emph{Appendix} 5.1D.

\hypertarget{e-guidance-regarding-appropriate-accommodations}{%
\paragraph{5.1E Guidance Regarding Appropriate
Accommodations}\label{e-guidance-regarding-appropriate-accommodations}}

Guidance regarding appropriate accommodations is published in
\emph{Appendix} 2.3A.1. District and School Test Coordinators provide
annual training on test security and administration. The ORExt
approaches access as part of test design, as noted above in Section
5.1D. The complexity of SWSCD communication systems demands such an
approach. In addition, comprehensive accommodations are allowed in order
to decrease the chances that a disability may interfere with our ability
to measure the student's knowledge and skills.

\hypertarget{f-all-swds-eligible-for-the-orext}{%
\paragraph{5.1F All SWDs Eligible for the
ORExt}\label{f-all-swds-eligible-for-the-orext}}

ODE's eligibility guidelines make it clear that all SWDs are eligible
for the ORExt, regardless of disability category, and that specific
disability category membership should not be a determining factor for
considering participation (see \emph{Appendix} 5.1B).

\hypertarget{g-parents-informed-of-aa-aaas-consequences}{%
\paragraph{5.1G Parents Informed of AA-AAAS
Consequences}\label{g-parents-informed-of-aa-aaas-consequences}}

The Parent FAQ section of the General Administration Manual makes it
clear that parents must be informed of the potential consequences of
having their child assessed against alternate achievement standards,
including diploma options. Parents are also informed that alternate
achievement standards are designed to reflect a significant reduction in
depth, breadth, and complexity and are therefore not comparable to
general academic achievement standards (see \emph{Appendix} 2.3,
p.~28-32).

\hypertarget{h-state-ensures-orext-promotes-access-to-the-general-education-curriculum}{%
\paragraph{5.1H State Ensures ORExt Promotes Access to the General
Education
Curriculum}\label{h-state-ensures-orext-promotes-access-to-the-general-education-curriculum}}

The ORExt is strongly linked to the CCSS/ORSci/NGSS, as evidenced by our
linkage study results (see \emph{Appendix} 3.1A). The claim is based on
the following warrants: (a) ORExt items are aligned to the Essentialized
Standards; (b) the Essentialized Standards are strongly linked to the
grade level content standards; therefore (c) the ORExt items are
strongly linked to grade level content expectations. It is thus expected
that the ORExt promotes access to the general education curriculum by
assessing general education content that has been reduced in depth,
breadth, and complexity yet maintains the highest possible standard for
SWSCD.

In addition, ODE commissioned BRT to work with Oregon teachers of SWSCD
in the 2015-16 school year to develop a variety of curricular and
instructional resources that are aligned to the Essentialized Standards.
These resources include: (a) curricular templates, (b) video tutorials,
and (c) supporting documents that provide specific guidance regarding
how to develop lesson plans, Present Levels of Academic and Functional
Performance (PLAAFP) statements, and Individualized Education Program
(IEP) goals and objectives that are aligned with the Essentialized
Standards. It is also expected that the essentialization process will
generalize to many students who are performing off grade level, not
merely to SWSCD. All resources are published on a
\color{link}\href{http://lms.brtprojects.org}{BRT-sponsored
website}\color{black}.

\hypertarget{a---5.2c-procedures-for-including-els}{%
\subsubsection{5.2A - 5.2C Procedures for Including
ELs}\label{a---5.2c-procedures-for-including-els}}

In addition to the programmatic guidance provided in \emph{Appendix}
1.4A.1 related to EL program eligibility and services, ODE also provides
guidance relevant to the inclusion of ELs in the statewide assessment
program in \emph{Appendix} 1.4.2. Though the ORExt is currently
published in English, an appropriately qualified interpreter can provide
the assessment to any SWSCD from diverse language backgrounds, including
American Sign Language. ODE has developed a training module to increase
the standardization of \color{link}\href{http://lms.brtprojects.org}{ASL
administration} \color{black} for its statewide assessments.

Additional information regarding the inclusion of ELs in Oregon's
general assessments is provided in the general assessment Peer Review
evidence.

\hypertarget{accommodations}{%
\subsubsection{5.3 Accommodations}\label{accommodations}}

All statewide accommodation guidance is published in the Accessibility
Manual (see \emph{Appendix} 2.3A.1), outlining the universal tools and
designated supports available to all students, and accommodations,
available only to students with disabilities or students served by
Section 504 Plans. In addition, the manual defines the supports as
embedded, where they are provided by the online test engine (e.g.,
calculator, text-to-speech), or non-embedded, where they must be
provided by a qualified assessor (e.g., read aloud, scribe). The manual
also makes it clear that these supports are content-area specific, as a
universal tool in one content area may be an accommodation in another.

\hypertarget{a-appropriate-accommodations-are-available-for-swd-section-504}{%
\paragraph{5.3A Appropriate Accommodations are Available for SWD/
Section
504}\label{a-appropriate-accommodations-are-available-for-swd-section-504}}

Appropriate accommodations for the ORExt are published in
\emph{Appendix} 2.3A.1, p.~36-43. Additional accommodations for all
statewide assessments are also published in this manual. The Oregon
Accommodations Panel reviews the appropriateness of the supports listed
annually. Practitioners may also request the addition of an
accommodation through a formal process (see \emph{Appendix} E: Approval
Process for New Accessibility Supports within the manual,
\emph{Appendix} 2.3A.1, p.~100-102).

\hypertarget{b-appropriate-accommodations-are-available-for-els}{%
\paragraph{5.3B Appropriate Accommodations are Available for
ELs}\label{b-appropriate-accommodations-are-available-for-els}}

As noted in Sections 5.2A-C, the ORExt is accessible in any
communication modality through the use of an interpreter. Appropriate
accommodations for the ORExt are published in \emph{Appendix} 2.3A.1,
p.~36-43. Additional accommodations for all statewide assessments are
also published in this manual. The Oregon Accommodations Panel reviews
the appropriateness of the supports listed annually. Practitioners may
also request the addition of an accommodation through a formal process
(see \emph{Appendix} E: Approval Process for New Accessibility Supports
within the manual, \emph{Appendix} 2.3A.1, p.~100-102).

\hypertarget{c-accommodations-are-appropriate-and-effective}{%
\paragraph{5.3C Accommodations are Appropriate and
Effective}\label{c-accommodations-are-appropriate-and-effective}}

In addition to the evidence gathered during the linkage study (see
\emph{Appendix} 3.1A), which suggests that the ORExt items were
accessible and free of bias even before final editing, the
appropriateness of the supports listed in \emph{Appendix} 2.3A.1 is
reviewed annually by the Oregon Accommodations Panel. Practitioners may
also request the addition of an accommodation through a formal process
(see \emph{Appendix} E: Approval Process for New Accessibility Supports
within the manual, \emph{Appendix} 2.3A.1, p.~100-102). ODE is
collecting accommodations codes for the ORExt from Qualified Assessors
who opt to enter this information in order to make performance
comparisons feasible. Accommodations information was collected in this
year's assessment. A study on the effect of the use of different
accommodations will be conducted and reported in the 2018-19 technical
report.

\hypertarget{d-accommodations-are-appropriate-and-effective}{%
\paragraph{5.3D Accommodations are Appropriate and
Effective}\label{d-accommodations-are-appropriate-and-effective}}

ODE has a formal process stakeholders can use to request accommodations
that are not already published in the Accessibility Manual (see
\emph{Appendix} E: Approval Process for New Accessibility Supports
within the manual, \emph{Appendix} 2.3A.1, p.~100-102).

\hypertarget{a---5.4e-monitoring-test-administration-for-special-populations}{%
\subsubsection{5.4A - 5.4E Monitoring Test Administration for Special
Populations}\label{a---5.4e-monitoring-test-administration-for-special-populations}}

ODE monitoring of test administration in its districts and schools is
elaborated within the general assessment Peer Review evidence and is
therefore not addressed here.


\end{document}
